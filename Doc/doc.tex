\documentclass[12pt, a4paper]{article}

\usepackage{amsmath}
\usepackage{amsfonts}
\usepackage{amssymb}
\usepackage{graphicx}
\usepackage{float}
\usepackage{listings}
\usepackage{rotating}
\usepackage{tikz}
\usepackage{verbatim}
\pdfgentounicode=1
\pdfmapline{+cyberb@Unicode@  <cyberbit.ttf}

\begin{document}

\title{PBMath}
\author{P. Baillehache}
\date{\today}
\maketitle

\tableofcontents

\section*{Introduction}

PBMath is C library providing mathematical structures and functions.\\ 

The \begin{ttfamily}VecFloat\end{ttfamily} structure and its functions can be used to manipulate vectors of float values.\\

The \begin{ttfamily}VecShort\end{ttfamily} structure and its functions can be used to manipulate vectors of short values.\\

The \begin{ttfamily}MatFloat\end{ttfamily} structure and its functions can be used to manipulate matrices of float values.\\

The \begin{ttfamily}Gauss\end{ttfamily} structure and its functions can be used to get values of the Gauss function and random values distributed accordingly with a Gauss distribution.\\

The \begin{ttfamily}Smoother\end{ttfamily} functions can be used to get values of the SmoothStep and SmootherStep functions.\\

The \begin{ttfamily}EqLinSys\end{ttfamily} structure and its functions can be used to solve linear equation systems.\\

\section{Definitions}

\subsection{Vectors}

\subsubsection{Distance between two vectors}

For \begin{ttfamily}VecShort\end{ttfamily}:\\

\begin{equation}
\begin{array}{l}
Dist(\overrightarrow{v},\overrightarrow{w})=\sum_i|v_i-w_i|\\
HamiltonDist(\overrightarrow{v},\overrightarrow{w})=\sum_i|v_i-w_i|\\
PixelDist(\overrightarrow{v},\overrightarrow{w})=\sum_i|v_i-w_i|\\
\end{array}
\end{equation}
For \begin{ttfamily}VecFloat\end{ttfamily}:\\
\begin{equation}
\begin{array}{l}
Dist(\overrightarrow{v},\overrightarrow{w})=\sum_i(v_i-w_i)^2\\
HamiltonDist(\overrightarrow{v},\overrightarrow{w})=\sum_i|v_i-w_i|\\
PixelDist(\overrightarrow{v},\overrightarrow{w})=\sum_i\left|\lfloor v_i\rfloor -\lfloor w_i\rfloor\right|\\
\end{array}
\end{equation}


\subsubsection{Angle between two vectors}

The problem is as follow: given two vectors $\vec{V}$ and $\vec{W}$ not null, how to calculate the angle $\theta$ from $\vec{V}$ to $\vec{W}$.\\

Let's call $M$ the rotation matrix: $M\vec{V}=\vec{W}$, and the components of $M$ as follow:
\begin{equation}
M=\left[
\begin{array}{cc}
Ma&Mb\\
Mc&Md\\
\end{array}
\right]=\left[
\begin{array}{cc}
cos(\theta)&-sin(\theta)\\
sin(\theta)&cos(\theta)\\
\end{array}
\right]
\end{equation}
Then, $M\vec{V}=\vec{W}$ can be written has 
\begin{equation}
\left\lbrace
\begin{array}{l}
W_x = M_aV_x+M_bV_y\\
W_y = M_cV_x+M_dV_y\\
\end{array}
\right.
\end{equation}
Equivalent to
\begin{equation}
\left\lbrace
\begin{array}{l}
W_x = M_aV_x+M_bV_y\\
W_y = -M_bV_x+M_aV_y\\
\end{array}
\right.
\end{equation}
where $M_a=cos(\theta)$ and $M_b=-sin(\theta)$.\\
If $Vx\neq0.0$, we can write
\begin{equation}
\left\lbrace
\begin{array}{l}
M_b = \frac{M_aV_y-W_y}{V_x}\\
M_a = \frac{W_x+W_yV_y/V_x}{V_x+V_y^2/V_x}\\
\end{array}
\right.
\end{equation}
Or, if $Vx=0.0$, we can write
\begin{equation}
\left\lbrace
\begin{array}{l}
Ma = \frac{W_y+M_bV_x}{V_y}\\
Mb = \frac{W_x-W_yV_x/V_y}{V_y+V_x^2/V_y}\\
\end{array}
\right.
\end{equation}
Then we have $\theta=\pm cos^{-1}(M_a)$ where the sign can be determined by verifying that the sign of $sin(\theta)$ matches the sign of $-M_b$: if $sin(cos^{-1}(M_a))*M_b > 0.0$ then multiply $\theta=-cos^{-1}(M_a)$ else $\theta=cos^{-1}(M_a)$.

\section{Interface}

\begin{scriptsize}
\begin{ttfamily}
\verbatiminput{../pbmath.h}
\end{ttfamily}
\end{scriptsize}

\section{Code}

\begin{scriptsize}
\begin{ttfamily}
\verbatiminput{../pbmath.c}
\end{ttfamily}
\end{scriptsize}

\section{Makefile}

\begin{scriptsize}
\begin{ttfamily}
\verbatiminput{../Makefile}
\end{ttfamily}
\end{scriptsize}

\section{Usage}

\begin{scriptsize}
\begin{ttfamily}
\verbatiminput{../main.c}
\end{ttfamily}
\end{scriptsize}

Output:\\
\begin{scriptsize}
\begin{ttfamily}
\verbatiminput{../output.txt}
\end{ttfamily}
\end{scriptsize}

vecshort.txt:\\
\begin{scriptsize}
\begin{ttfamily}
\verbatiminput{../vecshort.txt}
\end{ttfamily}
\end{scriptsize}

vecfloat.txt:\\
\begin{scriptsize}
\begin{ttfamily}
\verbatiminput{../vecfloat.txt}
\end{ttfamily}
\end{scriptsize}

matfloat.txt:\\
\begin{scriptsize}
\begin{ttfamily}
\verbatiminput{../matfloat.txt}
\end{ttfamily}
\end{scriptsize}

smoother functions:\\
\begin{center}
\begin{figure}[H]
\centering\includegraphics[width=6cm]{./smoother.png}\\
\end{figure}
\end{center}

gauss function (mean:0.0, sigma:1.0):\\
\begin{center}
\begin{figure}[H]
\centering\includegraphics[width=6cm]{./gauss.png}\\
\end{figure}
\end{center}

gauss rand function (mean:1.0, sigma:0.5):\\
\begin{center}
\begin{figure}[H]
\centering\includegraphics[width=6cm]{./gaussrnd.png}\\
\end{figure}
\end{center}

\end{document}


